%%%%%%%%%%%%%%%%%%%%%%%%%%%%%%%%%%%%%%%%%%%%%%%%%%%%%%%%%%%%%%%%%%%%%%%%%%%%%%%%%%%%%
%
%
%
%
%%%%%%%%%%%%%%%%%%%%%%%%%%%%%%%%%%%%%%%%%%%%%%%%%%%%%%%%%%%%%%%%%%%%%%%%%%%%%%%%%%%%%
\documentclass[12pt]{article}
%
\usepackage[a4paper]{geometry}
\usepackage[slovene,english,german]{babel}
\usepackage[T1]{fontenc}
\usepackage[cp1250]{inputenc}
\usepackage{graphicx}
\addtolength{\textheight}{1cm}
%
%

\author{Jaka �op}
\title{Izleti v istrske vasi (predstavitev)}
%
\begin{document}
\selectlanguage{slovene}
\pagenumbering{gobble}
  \maketitle
  
�stra (italijansko Istria) je polotok v jugozahodni Sloveniji in severozahodni Hrva�ki na severu Jadranskega morja. Majhen del sega tudi v Italijo. Prebivalci Istre so Istrani.

Istra je tradicionalno etni�no me�ana in jo sestavljajo Hrvati, Slovenci, Italijani in drugi. V Istri so zastopani hrva�ki jezik (�akav��ina), slovenski in italijanski (istrobene�ki) jezik. Manj�e �tevilo prebivalcev na posameznih obmo�jih pa govori �e istroromunski (t. i. vla�ki in �ejanski) ter romanski istriotski jezik. Ve�ina prebivalstva je dvo ali ve�jezi�na.

Istra je najve�ja zelena oaza severnega Jadrana. Vzdol� obale in na otokih prevladujejo borovi gozdovi in prepoznavna zelena makija. Glavna predstavnika makije sta �rnika in jagodi�nica. 

Slovenska Istra kljub svoji majhnosti ponuja ne�teto mo�nosti za odli�en izlet. Naj si bodo to ve�ja mesta ob obali kot so Koper, Piran in Portoro�, ali pa ne�teto manj�ih vasic kot so: Se�ovlje, Padna, Hrastovlje, Dragonja, Krkav�e ter mnoge druge. V vsaki istrski vasici je mo� najti kak�no zanimivost: od solin, galerije Bo�idarja Jakca, pa vse do taborske cerkvice v Hrastovljah katere fresko Mrtva�ki ples poznamo vsi. Poleg �tevilnih znamenitosti, ki jih je mo� obiskati v Istri pa se bodo vsi, ki so Istro kdaj obiskali strinjali, da je to predvsem kraj za sprostitev in oddih. Prav tako pa ob bo�jem miru istrske vasice predstavljajo tudi raj za ornitologe in botanike s svojo izjemno obse�no naravno floro in favno pa tudi z veliko raznolikostjo �ivalskega sveta.

Za konec pa �e: zamislite si pe��ico tradicionalnih istrskih kamnitih hi�k, temu dodajte valovite gri�ke, polja in nasade oljk - in tu je slika istrske vasi.

  
  
  
  
\end{document}