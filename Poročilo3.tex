%%%%%%%%%%%%%%%%%%%%%%%%%%%%%%%%%%%%%%%%%%%%%%%%%%%%%%%%%%%%%%%%%%%%%%%%%%%%%%%%%%%%%
%
%
%
%
%%%%%%%%%%%%%%%%%%%%%%%%%%%%%%%%%%%%%%%%%%%%%%%%%%%%%%%%%%%%%%%%%%%%%%%%%%%%%%%%%%%%%
\documentclass[12pt]{article}
%
\usepackage[a4paper]{geometry}
\usepackage[slovene,english,german]{babel}
\usepackage[T1]{fontenc}
\usepackage[cp1250]{inputenc}
\usepackage{graphicx}
\usepackage{bchart}
\usepackage{fancyhdr}
\usepackage{gensymb}
\usepackage{wrapfig}
\usepackage[export]{adjustbox}
\usepackage{amsmath}
\addtolength{\textheight}{1cm}
%
%

\author{Jaka �op}
\title{Terenska vaja - dru�benogeografska \\ Gospodarska funkcija Strunjanskega zaliva}
%
\begin{document}
\selectlanguage{slovene}
\pagenumbering{gobble}
  \maketitle

Na vajo smo se kot pri prvi delno pripravili �e v �oli, ker so priprave potekale zelo podobno kot pri pri vaji jih tu ne bom podrobno opisoval.

Po prihodu v Strunjan smo se razdelili v dve skupini. Sam sem bil v skupini, ki je najprej opravila obe naravnogeografski vaji. Ko smo opravili obe vaji smo se zamenjali z drugo skupino in se odpravili opraviti prvo dru�benogeografsko vajo in sicer vajo z naslovom: gospodarska funkcija Strunjanskega zaliva. Za to vajo smo potrebovali le nekaj barvic trdo podlago ter �rno-belo karto Strunjanskega zaliva. Ko sem se odpravil proti Strunjanu, sem najprej naletel na ob�irno karto celotnega obmo�ja in ugotovil, da se bistveno razlikuje od karte, ki sem jo dr�al v rokah, zato si z prvo karto nisem mogel kaj dosti pomagati in sem se odpravil pe� po obmo�ju. Nekaterih podatkov, ki jih je vaja zahtevala nisem na�el in sicer: razpadajo�ih zgradb in zgradb v gradnji. To je seveda logi�no glede na to, da je bila takrat �e turisti�na sezona in gradnje takrat niso smiselne zaradi morebitne izgube turistev. Kar pa se ti�e razpadajo�ih zgradb je bilo to zame manj�e presene�enje, ali pa tudi ne saj je Strunjan turisti�no mesto, turiste pa z podrtimi zgradbami te�ko privabi�.

Najpomembnej�a gospodarksa funkcija Strunjanskega zaliva je brez dvoma turizem, kar potrjuje tudi rezultat kartiranja, saj je v zalivu najve� stanovalskih objektov ter objektov s turisti�no funkcijo. Najve� je apartmajev, hotelov, restavracij, kafi�ev, rekreacijskih povr�in ter parkiri��. Mo�nosti za nadaljni gospodarski razvoj je zagotovo veliko, osebno se mi zdi najbolj smiselno vlaganje v �e obsotoje�o gospodarsko funkcijo to je turizem, najpomembnej�e je, da se zdaj�nje povr�ine ohranjajo in posoddabljajo, tako da bi lahko morda neko� konkurirali celo turisti�no bolj razvitim mestom. Seveda so tu tudi druge mo�nosti uredbe. Na primer okrog stju�e bi lahko postavili rekreacijsko pot s fitnes pripomo�ki, podobno tisti ob Ve�ni poti v Ljubljani.



\end{document} 